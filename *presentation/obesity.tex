\documentclass{philslides}

\usepackage[authordate,isbn=false,backend=biber,cmsdate=new, noibid]{biblatex-chicago}
	%\addbibresource{other-bib-file}
	\addbibresource{library.bib}
	\appto{\citesetup}{\tiny}

\usetheme{Frankfurt}
\usetikzlibrary{positioning}
\beamertemplatenavigationsymbolsempty

\AtBeginSection[]{
	\frame
	{
		\frametitle{Outline}
		\tableofcontents[currentsection]
	}
}

\newcommand{\talkurl}{\href{}{}}

\begin{document}
\date{}
\title{Many Models and Public Health Data}
\author[Dan Hicks]{
	Dan Hicks (Data Science Initiative, UC Davis)\\
	{\scriptsize \href{mailto:djhicks@ucdavis.edu}{\texttt{djhicks@ucdavis.edu}}}\\ 
	{\scriptsize \href{https://twitter.com/danieljhicks}{\texttt{@danieljhicks}}}\\
%	{\scriptsize This talk: \talkurl}
	Catherine Womack (Bridgewater State University)
	}
	
\frame
{
	\titlepage
}

\frame
{
	\frametitle{The Whole Talk in One Slide}
	\begin{center}
	\parbox{.67\textwidth}{
		What happens when you run 20 different\\ regression models on the same data?  
	}
	\end{center}
}


\frame
{
	\frametitle{Outline}
	\tableofcontents
}


\section{Background}
\subsection{}

\frame
{
	\frametitle{Background:  DJH}
	\begin{itemize}
	\item Philosophy of science, STS
	\item Public scientific controversies
		\begin{itemize}
		\item Genetically modified foods \autocite{Hicks2015, Hicks2016a, Hicks2017b}
		\item Climate \autocite{Hicks2017a}
		\item Vaccines \autocite{Hicks2017a}
		\item \blue{Obesity}
		 \end{itemize}
	\end{itemize}
}
\frame
{
	\frametitle{Background: Obesity Research}
	\begin{itemize}
	\item Epidemiology and public health
		\begin{itemize}
		\item \textbf{Neither of us have any training in either field}
		\item \textbf{Nothing I say here should be construed as medical advice}
		\end{itemize}
	\item ``Fatness'' operationalized as \red{body mass index} [BMI]
		\begin{itemize}
		\item Continuous variable:  $\displaystyle \frac{weight}{height^2}$
		\item \red{Usually broken into discrete categories}:\\
			underweight, normal, overweight, obese I, obese II
		\end{itemize}
	\item Frequently modeled with Cox proportional hazards
	\end{itemize}
}
%\frame
%{
%	\frametitle{Background: ``Obesity Paradox''}
%	\makebox[\textwidth]{
%		\includegraphics[width = 1.1\textwidth]{Flegal-chart.png}
%	}
%	\vfill
%	\cites{Flegal2005}[see also][]{McAuley2012; Kramer2013}
%%	\vfill
%%	\begin{center}
%%		\includegraphics[height = .33\textheight]{McAuley-fig1a.pdf}
%%	\end{center}
%%	\cite{McAuley2012}
%}



\section{Many Models}
\subsection*{}
\frame
{
	\frametitle{Dataset}
	\begin{itemize}
		\item NHANES III (1988-1994) + 1999-2004\\
			{\tiny \url{https://www.cdc.gov/nchs/nhanes/index.htm}}
		\item Linked mortality data from 2011
		\item Complex survey design \autocite{Lumley2010}
		\item My dataset includes only individuals who
		\begin{itemize}
			\item never smoked
			\item BMI $<$ 75
			\item age $\geq$ 50, age $<$ 85 when participated in NHANES
			\item survived $\geq 1$ month after participating
			\item have data for BMI, education level, followup mortality
		\end{itemize}
		\item Total 5,677 individuals
	\end{itemize}
}

\frame
{
	\frametitle{Many Models}
	\begin{center}
	\begin{tabular}{c|c}
	\textbf{BMI Specification} & \textbf{Model Specification}\\
	\hline
	Binned or discrete BMI & Linear \\
	Continuous BMI & Logistic\\
	Square BMI & Poisson\\
	4-knot spline & Cox PH\\
	6-knot spline &\\
	\end{tabular}
	\end{center}
	Total $5 \times 4 = 20$ regression models
}
\frame
{
	\frametitle{BMI Specifications}
	\begin{description}
	\item[Binned] BMI is divided into discrete bins or categories:  
		\begin{center}
		\small
		\begin{tabular}{cl}
		$\leq 18.5$ & underweight\\
		18.5-25 & normal weight\\
		25-30 & overweight\\
		30-35 & obese I\\
		$\geq 35$ & obese II
		\end{tabular}
		\end{center}
	\item[Continuous] $bmi$
	\item[Square] $bmi + bmi^2$
%	\item[Cubic] $bmi + bmi^2 + bmi^3$
	\item[Splines] 
		\begin{itemize}
		\scriptsize
		\item Range of BMI values is divided into $k+1$ subintervals at $k$ knots.  
		\item Separate polynomials (cubics) are fit to each subinterval.  
		\item Polynomials are constrained to be equal at knots.  
		\item 4 knots at quintiles (20\%, 40\%, \ldots) of BMI distribution
		\item 6 knots at septiles (1/7, 2/7, \ldots) of BMI distribution
		\end{itemize}
	\end{description}
}
\frame
{
	\frametitle{Model Specifications}
	\begin{description}
	\item[Linear] \blue{$mortality$ is a continuous, unbounded variable}
		\[mortality \sim Gauss(\mu, \sigma); g(\mu) = \mu\]
	\item[Logistic] \blue{$mortality$ is a binary outcome (viz., dying or not)}
		\[mortality \sim Bernoulli(\theta); g(\theta) = \log \frac{\theta}{1-\theta}\]
	\item[Poisson] \blue{$mortality$ is a count of events (viz., number of times the individual dies)}
		{\tiny (\url{https://stats.stackexchange.com/questions/18595/})}
		\[mortality \sim Poisson(\lambda); g(\lambda) = \log(\lambda)\]
	\item[Cox PH] \blue{Response is hazard ratio at time $t$}: 
		\[\frac{\eta(t | \mathbf{X})}{\eta_0(t)} = e^{\mathbf{X\beta}} \]
	\end{description}
}
\frame
{
	\frametitle{Model Outputs: RR Predictions}
	\makebox[\textwidth]{
	\includegraphics[width = 1.2\textwidth]{../02_rr_preds.png}
	}
}
\frame
{
	\frametitle{Model Outputs: RR Predictions}
	\makebox[\textwidth]{
	\includegraphics[width = 1.2\textwidth]{../03_rr_preds.png}
	}
}
\frame
{
	\frametitle{Note}
	\begin{itemize}
	\item Cox PH is doing something completely different.  
	\item Other model specifications generally agree with each other.  
	\item BMI specification makes a dramatic difference.  
	\end{itemize}
}



\section{Model Selection}
\subsection*{}
\frame
{
	\frametitle{Model Selection to the Rescue?}
	\begin{itemize}
	\item Q: What happens when you run the same data through 20 different models?  
	\item A: They disagree!  
	\item But:  We can use model selection techniques (?)
	\end{itemize}
}
\frame
{
	\frametitle{Model Selection: The Common Strategy}
	\begin{enumerate}
	\item Construct all your models
	\item Calculate \red{evaluation statistics} for each model
		\begin{itemize}
		\item AIC
%			\begin{itemize}
%			\item AIC can be used with non-nested models *[cite]
%			\end{itemize}
		\item For continuous response:  $R^2$, RSS, RMSE, \ldots.
		\item For binary response:  Accuracy, sensitivity (FNR), precision, $\kappa$, AUROC, \ldots.
			\begin{itemize}
			\item With or without CV
			\end{itemize}
		\item ANOVA + F
		\item \ldots
		\end{itemize}
	\item Select the model with the best statistic
	\end{enumerate}
}
\frame
{
	\frametitle{But}
	\makebox[\textwidth]{
		\includegraphics[width = 1\textwidth]{../01_pred_fit.png}
	}
}


\section{Questions}
\subsection*{}
\frame
{
	\frametitle{Questions}
	\begin{enumerate}
	\item Can AIC be used across different model specifications? 
	\item What other methods are used in your field for model selection?  
	\item Instead of selecting one best model, what methods are used in your field to aggregate many models?  
	\end{enumerate}
}




\section*{}
\frame
{
	\titlepage
}

\frame[allowframebreaks]
{
	\setbeamertemplate{bibliography item}{}
	%\renewcommand{\bibfont}{\fontsize{2}{3}\selectfont }
	\renewcommand{\bibfont}{\tiny}
	\renewcommand{\bibitemsep}{-.2em}
	\printbibliography[heading=none]
}

\end{document}